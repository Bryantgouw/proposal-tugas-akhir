% ==========================================
% BAB I PENDAHULUAN
% ==========================================
\chapter{PENDAHULUAN}
\label{chap:pendahuluan}
% --- Latar Belakang ---
\section{Latar Belakang}
Sampoerna Retail Company (SRC) merupakan sebuah \textit{development program} yang diinisiasi oleh PT HM Sampoerna Tbk untuk peritel tradisional, atau lebih dikenal dengan toko kelontong. Program ini bertujuan untuk meningkatkan daya saing toko kelontong melalui praktik bisnis yang berkelanjutan. Upaya tersebut sejalan dengan misi SRC untuk mentransformasi toko kelontong yang sebelumnya tampak gelap dan kurang tertata menjadi lebih bersih, rapi, dan terang. Dalam mewujudkan misinya, SRC memberikan dukungan yang komprehensif bagi para anggotanya untuk berkembang menjadi peritel yang lebih baik, modern, dan adaptif terhadap perkembangan teknologi, khususnya dalam digitalisasi \cite{Sutanto2024}. 

Selama keberjalanan program, omzet Toko SRC berhasil menyumbang sekitar Rp 263 triliun per tahun terhadap perekonomian indonesia, setara dengan 11,36\% dari total Produk Domestik Bruto (PDB) Ritel Nasional. Selain itu, proses digitalisasi yang dilakukan SRC di tengah perkembangan ekonomi digital mampu meningkatkan rata-rata omzet Toko SRC hingga 42\% \cite{KompasGramediaMedia2023}.
Di balik pencapaian tersebut, keberhasilan SRC didukung oleh ekosistem AYO by SRC. 
\textcite{SampoernaRetailCompany2025} menjelaskan bahwa ekosistem ini terbagi atas tiga segmen utama, yaitu mitra, toko, dan pelanggan. Mitra merupakan pedagang grosir yang berperan dalam memastikan ketersediaan pasokan produk bagi Toko SRC dan ritel lainnya dengan dukungan aplikasi AYO Mitra. Toko SRC sebagai pengguna aplikasi AYO TOKO hadir untuk menyediakan kebutuhan harian masyarakat Indonesia. Sementara itu, pelanggan sebagai pengguna barang dan jasa Toko SRC dapat melakukan pembelian secara daring melalui aplikasi MY AYO. 

AYO TOKO by SRC menjadi salah satu aplikasi kunci dalam menjaga keberlangsungan ekosistem digital yang dibangun SRC. Meskipun sekitar 90\% Toko SRC telah mengadopsi aplikasi AYO TOKO, hanya 5 dari 10 peritel yang menggunakannya secara aktif setiap bulan \cite{INKOMPASS2024}.
Menurut Bapak \textcite{Junaidi2025}, eksekutif dari Divisi \textit{Data Intelligence and Analytics} di PT HM Sampoerna Tbk, jumlah \textit{daily active users} dan frekuensi penggunaan aplikasi AYO TOKO mengalami penurunan. Berdasarkan pengamatannya, salah satu penyebab kondisi tersebut adalah kebiasaan sebagian pemilik Toko SRC yang mayoritas merupakan individu dewasa madya dan masih mengandalkan \textit{salesman} SRC untuk membantu melakukan pemesanan barang secara manual. 

Penelitian dari \textcite{Chou2007} menunjukkan bahwa populasi dewasa madya memiliki karakteristik unik dibandingkan kategori usia lainnya dalam proses belajar dan penggunaan komputer. Hasil penemuannya menunjukkan bahwa desain antarmuka saat ini sering kali tidak mudah dipahami secara fungsional, sehingga menyulitkan bagi dewasa madya untuk membiasakan diri dalam penggunaan perangkat. 
Selain itu, \textcite{JosephND} juga menemukan bahwa kelompok dewasa madya (40-59 tahun) mulai menunjukkan penurunan performa yang signifikan, terutama dalam penyelesaian tugas yang kompleks. Dengan demikian, diperlukan penelitian yang lebih mendalam untuk menganalisis faktor-faktor yang berkontribusi terhadap rendahnya tingkat penggunaan aplikasi AYO TOKO, khususnya di kalangan mayoritas penggunanya yang berada pada kelompok dewasa madya. Hasil analisis ini diharapkan dapat menjadi landasan dalam merancang solusi yang berfokus pada peningkatan kualitas desain interaksi dan pengalaman pengguna aplikasi AYO TOKO yang inklusif, khususnya bagi pengguna yang mayoritas dewasa madya. 

% --- Rumusan Masalah ---
\section{Rumusan Masalah}
Berdasarkan latar belakang yang telah dijelaskan pada subbab sebelumnya, penelitian ini difokuskan pada perancangan desain interaksi yang inklusif untuk menjawab tantangan dan kebutuhan pengguna aplikasi AYO TOKO yang mayoritas dewasa madya, sehingga pengguna dapat merasakan relevansi dan manfaat nyata dari penggunaan aplikasi tersebut. Adapun rumusan masalah yang didefinisikan sebagai berikut : 
\begin{enumerate}
\item	Bagaimana karakteristik demografi dari pemilik Toko SRC di Kota Bandung?
\item	Apa tantangan yang dihadapi pemilik Toko SRC saat berinteraksi dengan aplikasi AYO TOKO?
\item	Bagaimana perancangan desain interaksi yang inklusif dapat meningkatkan aksesibilitas dan kemudahan penggunaan aplikasi AYO TOKO bagi pemilik Toko SRC?
\end{enumerate}

% --- Tujuan ---
\section{Tujuan}
Berdasarkan rumusan masalah yang telah ditetapkan, penelitian ini memiliki tujuan untuk : 
\begin{enumerate}
\item	Mengidentifikasi karakteristik demografi dari pemilik Toko SRC di Kota Bandung.
\item   Menganalisis tantangan dan kebutuhan pemilik Toko SRC dalam menggunakan aplikasi AYO TOKO. 
\item   Merancang solusi desain interaksi yang lebih aksesibel dan mudah untuk digunakan bagi pengguna aplikasi AYO TOKO. 
\item   Mengevaluasi solusi yang dibuat dapat meningkatkan \textit{usability} dan mendukung pencapaian beberapa \textit{UX goals}. 
\end{enumerate}

% --- Batasan Masalah ---
\section{Batasan Masalah}
Penyusunan penelitian ini membuat peneliti menyadari bahwa ruang lingkup permasalahan yang diangkat cukup luas. Dengan demikian, diperlukan batasan masalah yang telah dijabarkan sebagai berikut : 
\begin{enumerate}
\item	Penelitian ini difokuskan pada pengembangan aplikasi AYO TOKO, tanpa membahas aplikasi pendukung lainnya di dalam ekosistem AYO (AYO Mitra dan MY AYO).
\item   Analisis permasalahan pada aplikasi AYO TOKO dalam penelitian ini dibatasi hanya pada wilayah Kota Bandung. 
\item   Fokus penelitian ini terbatas pada peningkatan serta penyelesaian permasalahan desain interaksi pada aplikasi AYO TOKO, sehingga permasalahan yang bersifat teknis maupun implementasi sistem tidak akan diteliti lebih lanjut. 
\item   Beberapa fitur utama, yaitu Misi, Langganan, Pojok Bayar, Pojok Untung, Katalog SRC, dan Promosi, tidak tersedia untuk akses umum dan hanya dapat digunakan oleh anggota SRC. Oleh karena itu, penelitian ini dibatasi pada fitur yang dapat diakses oleh peneliti, seperti Belanja dan Akses Bantuan. 
\end{enumerate}

% --- Metodologi Pengerjaan TA ---
\section{Metodologi}
Penelitian ini menggunakan metodologi \textit{User-Centered Design} (UCD) sebagai kerangka utama untuk membuat rancangan dan evaluasi desain interaksi aplikasi AYO TOKO. Pemilihan metodologi ini dilandaskan pada karakteristik UCD yang berorientasi pada kebutuhan, ekspektasi, dan kendala yang dihadapi pengguna. Dengan menerapkan perspektif pengguna dalam setiap tahap rancangan, aplikasi dapat memberikan pengalaman yang lebih menyenangkan sehingga mendorong pengguna untuk terus menggunakannya dan memberikan rekomendasi kepada orang lain. Hal ini berpotensi untuk meningkatkan penjualan dari aplikasi serta menekan biaya yang disebabkan oleh revisi desain ataupun kebutuhan \textit{customer service}. Metodologi UCD terbagi ke dalam empat tahapan utama yaitu, \textit{understand context of use, specify user requirements, design solutions,} dan \textit{evaluate against requirements}. Setiap tahapan dalam penelitian ini dirancang secara terintegrasi dan berkesinambungan untuk menghasilkan solusi yang akurat dan dapat dipertanggungjawabkan.
\begin{enumerate}
\item	\textit{Understand Context of Use} \\
Penelitian diawali dengan menganalisis konteks penggunaan dari pengguna aplikasi AYO TOKO. Data tersebut diperoleh melalui wawancara langsung yang dilakukan oleh peneliti kepada sejumlah pemilik toko SRC di Kota Bandung. Wawancara mencakup pertanyaan mengenai umur pemilik toko, frekuensi penggunaan aplikasi, kendala yang dihadapi, serta beberapa pertanyaan lain yang relevan. Tahap ini menghasilkan pemetaan kebutuhan dan permasalahan pengguna yang menjadi dasar dalam penyusunan \textit{user requirements}. \\
\item   \textit{Specify User Requirements} \\
Berdasarkan hasil pemetaan kebutuhan dan permasalahan dari pengguna, peneliti menentukan \textit{requirements}—kebutuhan dan permasalahan apa yang menjadi prioritas untuk diselesaikan. Pada tahap ini, peneliti mengidentifikasi dua jenis kebutuhan, yaitu \textit{functional requirements} dan \textit{non-functional requirements}. Selain itu, peneliti juga mendiskusikannya dengan \textit{stakeholder} dari perusahaan PT HM Sampoerna Tbk untuk memastikan keselarasan antara kebutuhan pengguna dengan tujuan bisnis yang ingin dicapai perusahaan. \\
\item   \textit{Design Solutions} \\
Tahap ini berfokus pada perancangan solusi berdasarkan \textit{requirements} yang telah ditentukan pada tahap sebelumnya. Proses perancangan dilakukan dengan menghasilkan beberapa \textit{deliverables}, seperti \textit{user flow, wireframe, mock-up,} dan \textit{prototype}. Seluruh \textit{deliverables} dibuat dengan menggunakan \textit{platform} Figma untuk memastikan konsistensi dan kualitas tampilan antarmuka yang optimal. Selain itu, \textit{prototype} yang dihasilkan pada tahap ini berfungsi sebagai dasar untuk proses evaluasi pada tahap berikutnya. \\
\vspace*{1.5cm}
\item   \textit{Evaluate Against Requirements} \\
Setelah proses perancangan solusi, peneliti melanjutkan ke tahap evaluasi dengan menerapkan metode pengujian \textit{usability testing}. Tahapan ini bertujuan untuk mengidentifikasi berbagai permasalahan terkait \textit{usability} serta menemukan area yang memerlukan perbaikan. Pengujian dilakukan secara langsung di hadapan pemilik toko SRC sebagai pengguna aplikasi AYO TOKO. Meskipun merupakan tahapan yang terakhir, proses evaluasi dilakukan secara iteratif sebanyak dua kali untuk memastikan adanya \textit{continuous feedback} dan penyempurnaan berkelanjutan terhadap solusi yang sedang dikembangkan.  \\
\end{enumerate}